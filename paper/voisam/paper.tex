\documentclass{article}
\usepackage{algpseudocode}
\usepackage[ruled]{algorithm}
\usepackage{url}
\usepackage{framed}
\usepackage{amsfonts,amsmath,amsthm,amssymb}
\usepackage{graphicx}
\usepackage{url}
\usepackage{color}
\usepackage{geometry}

\geometry{margin=1.2in}

\newcommand {\mean} {\ensuremath {\mathop{\mathrm{mean}}}}
\newcommand {\median} {\ensuremath {\mathop{\mathrm{median}}}}
\newcommand {\N} {\ensuremath {\mathcal{N}}}
\newcommand {\IE} {\ensuremath {\mathbb{E}}}
\newcommand {\cov} {\ensuremath {\mathop{\mathrm{cov}}}}
\newcommand {\BEL} {\ensuremath {\mathop{\mathrm{BEL}}}}

\newtheorem{lemma}{Lemma}

\title{VOI-based Monte Carlo Sampling}
\author {David Tolpin, Solomon Eyal Shimony \\
Department of Computer Science, \\
Ben-Gurion University of the Negev, Beer Sheva, Israel \\
\{tolpin,shimony\}@cs.bgu.ac.il}

\begin{document}

\maketitle

\begin{abstract}
Upper bounds for the VOI are provided for pure exploration in the
Multi-armed Bandit Problem. Sampling policies based on the upper
bounds are suggested.
\end{abstract}

\section{Introduction and Definitions}

Taking a sequence of samples in order to minimize the
regret of a decision based on the samples is abstracted by the
{\em Multi-armed Bandit Problem.} In the Multi-armed Bandit problem
we have a set of $K$ arms. Each arm can be pulled multiple
times. When the $i$th arm is pulled, a random reward $X_i$ from an
unknown stationary distribution is returned.  The reward is bounded
between 0 and 1.

The simple regret of a sampling policy for the Multi-armed Bandit
Problem is the expected difference between the best expected reward
$\mu_*$ and the expected reward $\mu_j$ of the arm with the best sample mean
$\overline X_j=\max_i\overline X_i$:
\begin{equation}
\label{eq:simple-regret}
\IE[R]=\sum_{j=1}^K\Delta_j\Pr(\overline X_j=\max_i\overline X_i)
\end{equation}
where $\Delta_j=\mu_*-\mu_j$.
Strategies that minimize the simple regret are called pure exploration
strategies \cite{Bubeck.pure}. Principles of rational metareasoning
\cite{Russel.right} suggest that at each step the arm with the great
value of information (VOI) must be pulled, and the sampling must be
stopped and a decision must be made when no arm has positive VOI. 

To estimate the VOI of pulling an arm, either a certain 
distribution of the rewards should be assumed (and updated based on
observed rewards), or a distribution-independent bound on the VOI can be
used as the VOI estimate. In this paper, we use {\em concentration inequalities}
to derive distribution-independent bounds on the VOI.

\section{Some Concentration Inequalities}

Let $X_1, \ldots, X_n$ be i.i.d. random variables with values from $[0,1]$,
$X=\frac 1 n \sum_{i=1}^n X_i$. Then 
\begin{description}
\item[Hoeffding's inequality \rm{\cite{Hoeffding.ineq}}:] 
\begin{equation}
\Pr(X-\IE[X] \ge a) \le \exp ( -2na^2)
\label{eq:conc-hoeffding}
\end{equation}
\item[Empirical Bernstein's inequality
  \rm{\cite{MaurerPontil.benrstein}}:]\footnote{see
    Appendix~\ref{app:deriv-conc-empbernstein} for derivation}
\begin{eqnarray}
\Pr(X-\IE[X] \ge a) &\le& 2\exp \left( - \frac {na^2} {\frac {14} {3}
                          \frac {n} {n-1}a+2\overline\sigma_n^2}\right)\nonumber\\
                    &\le& 2\exp \left( - \frac {na^2} {10a+2\overline\sigma_n^2}\right)
\label{eq:conc-empbernstein}
\end{eqnarray}
where sample variance $\overline\sigma_n^2$ is
\begin{equation}
\overline\sigma_n^2=\frac 1 {n(n-1)} \sum_{1\le i < j\le n}(X_i-X_j)^2
\label{eq:sample-variance}
\end{equation}
\end{description}
Bounds (\ref{eq:conc-hoeffding}, \ref{eq:conc-empbernstein}) are symmetrical
around the mean. Bound~(\ref{eq:conc-empbernstein}) is tighter than
(\ref{eq:conc-hoeffding}) for small $a$ and $\overline\sigma_n^2$. 


\section{Upper Bounds on Value of Information}

The intrinsic VOI of pulling an arm is the expected decrease
in the regret compared to selecting an arm without pulling any arm at
all. The intrinsic VOI can be estimated as the intrinsic 
value of perfect information $\Lambda_i$ about the mean reward of the $i$th arm. Two
cases are possible:
\begin{itemize}
\item the arm ($\alpha$) with the highest sample mean is pulled, and the 
mean of the arm is lower than the sample mean the second-best arm ($\beta$);
\item another arm is pulled, and the mean of the arm is higher
than the current highest sample mean ($\alpha$).
\end{itemize}
\begin{eqnarray}
\Lambda_\alpha&=&\int_0^{\overline X_\beta}p(\IE[X_\alpha]=x)(\overline X_\beta-x)dx\nonumber\\
\Lambda_{i|i\ne\alpha}&=&\int_{\overline X_\alpha}^1p(\IE[X_i]=x)(x-\overline X_\alpha)dx
\end{eqnarray}
Using the concentration inequalities, $\Lambda_i$ can be bounded from
above as the probability that a different arm is selected times the
maximum possible increase in the reward:
\begin{eqnarray}
\Lambda_\alpha&\le&\Pr(\IE[X_\alpha]\le\overline X_\beta)\overline X_\beta\nonumber\\
\Lambda_{i|i\ne \alpha}&\le&\Pr(\IE[X_i]\ge\overline
X_\alpha)(1-\overline X_\alpha)
\label{eq:lambda-general-bounds}
\end{eqnarray}

The probabilities in bounds~(\ref{eq:lambda-general-bounds}) can be
bounded from above by the concentration
inequalities~(\ref{eq:conc-hoeffding}, \ref{eq:conc-empbernstein});
thus upper bounds on $\Lambda$ corresponding to each of the
concentration inequalities are obtained. For the Hoeffding's
inequality~(\ref{eq:conc-hoeffding}):
\begin{eqnarray}
\Lambda_\alpha&\le&\hat\Lambda_\alpha^h=\overline X_\beta \exp\left(-2n(\overline  X_\alpha - \overline X_\beta)^2\right)\nonumber\\
\Lambda_{i|i\ne \alpha}&\le&\hat\Lambda_{i|i\ne \alpha}^h=(1-\overline X_\alpha) \exp\left(-2n(\overline X_\alpha - \overline X_i)^2\right)
\label{eq:lambda-hoeffding-bounds}
\end{eqnarray}
For the empirical Bernstein's inequality~(\ref{eq:conc-empbernstein}):
\begin{eqnarray}
\Lambda_\alpha&\le&\hat\Lambda_\alpha^h=2\overline X_\beta \exp\left(-\frac {n(\overline X_\alpha - \overline X_\beta)^2} {10(\overline X_\alpha - \overline X_\beta)+2\overline\sigma_{\alpha,n}^2}\right)\nonumber\\
\Lambda_{i|i\ne \alpha}&\le&\hat\Lambda_{i|i\ne \alpha}^h=2(1-\overline X_\alpha) \exp\left(-\frac {n(\overline X_\alpha - \overline X_i)^2} {10(\overline X_\alpha - \overline X_i)+2\overline\sigma_{i,n}^2}\right)
\label{eq:lambda-bernstein-bounds}
\end{eqnarray}
and a tighter bound can be constructed as $\hat\Lambda^{h,b}=\min(\hat\Lambda^h,\hat\Lambda^b)$.


\section{VOI-based Sampling Control}

\subsection{Selection Criterion}

Following the principles of rational metareasoning, an arm with the highest upper bound $\hat
\Lambda$ on the perfect value of information should be pulled at each
step. This way, arms known to have a low VOI  would be pulled less frequently.

\subsection{Termination Condition}

The upper bounds~(\ref{eq:lambda-hoeffding-bounds}, \ref{eq:lambda-bernstein-bounds}) decrease exponentially with the
number of pulls $n$. When the upper bound of the VOI for all arms
becomes lower than a threshold $\lambda$, which can be chosen based on
resource constraints, the sampling should be stopped, and an arm should
be chosen.

\appendix

\section{Empirical Bernstein Inequality}
\label{app:deriv-conc-empbernstein}

Theorem~4 in~\cite{MaurerPontil.benrstein} states that
\[\Pr\left(\IE[X]-\overline X_n \ge \sqrt { \frac {2\overline\sigma_n^2 \ln 2/\delta} n } + \frac {7 \ln 2/\delta} {3(n-1)}\right)\le \delta,\]
Therefore
\[\Pr\left(\IE[X]-\overline X_n \ge \sqrt { \left(\frac {7 \ln 2/\delta} {3(n-1)}\right)^2+\frac {2\overline\sigma_n^2 \ln 2/\delta} n } + \frac {7 \ln 2/\delta} {3(n-1)}\right)\le
\delta.\]
$a=\sqrt { \left(\frac {7 \ln 2/\delta} {3(n-1)}\right)^2+\frac {2\overline\sigma_n^2 \ln 2/\delta} n } + \frac {7 \ln 2/\delta} {3(n-1)}$ is a root of
square equation
\[a^2-a\frac {14 \ln 2/\delta} {3(n-1)} -\frac {2\overline\sigma_n^2 \ln 2/\delta} n=0\]
which, solved for $\delta\triangleq\Pr(\IE[X]-\overline X_n\ge a)$,
gives
\[\Pr(\IE[X]-\overline X_n\ge a)\le 2\exp \left( - \frac {na^2} {\frac {14} {3} \frac {n} {n-1}a+2\overline\sigma_n^2}\right)\]
Other derivations, giving slightly different results, are possible.

\bibliographystyle{plain}
\bibliography{refs}

\end{document}
