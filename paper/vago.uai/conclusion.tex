\section{Conclusion}

The selection problem has numerous applications. This paper formalized the
problem as a belief-state MDP and proved some important properties of the 
resulting formalism. An application of the selection problem to control of
sampling was examined, and the insights provided by properties of the MDP 
led to approximate solutions that improve the state of the art. This was
shown in empirical evaluation both in ``flat" selection and when extending
the methods to game-tree search for the game of Go.

The methods proposed in the paper open up several new research
directions. The first is a better approximate solution of the MDP,
that should lead to even better flat sampling algorithms for
selection. A more ambitious goal is extending the formalism to
trees---in particular, achieving better sampling at non-root nodes,
for which the purpose of sampling differs from that at the root.
