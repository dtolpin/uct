\begin{abstract}
Sequential decision problems are often approximately solvable by
simulating possible future action sequences.  {\em Metalevel} decision
procedures have been developed for selecting {\em which} action
sequences to simulate, based on estimating the expected
improvement in decision quality that would result from any particular
simulation; an example is the recent work on using bandit algorithms
to control Monte Carlo tree search in the
game of Go.  In this paper we develop a theoretical basis for
metalevel decisions in the statistical framework of {\em selection problems}, 
arguing (as others have done) that this is more appropriate than the bandit
framework.  We derive a number of basic results applicable to 
Monte Carlo selection problems, including the first finite sampling
bounds for optimal policies in certain cases; we also provide a simple
counterexample to the intuitive conjecture that an optimal policy will
necessarily reach a decision in all cases.  We then derive heuristic
approximations in both Bayesian and distribution-free settings and
demonstrate their superiority to bandit-based heuristics in one-shot
decision problems and in Go.
\end{abstract}
