%% \section{On optimal policies for selection}

%% Define selection problem and metalevel MDP representation.

%% \note{Unsure about terminology here: selection problems, sampling problems, 
%% metalevel probability model, metalevel decision problem (conflicts with Markov decision process?).}

%% ============================
%% \subsection{Selection problems}
%% ============================

%% \note{Formal definition of selection problems and the metalevel MDP with cost per sample (time value); also mention budgeted learning.}

In a selection problem the decision maker is faced with a choice among
alternative arms\footnote{Alternative actions are known as \emph{arms} in the bandit setting;
we borrow this terminology for uniformity.}.  To make this choice, they may gather evidence about the
utility of each of these alternatives, at some cost.  The objective is to maximize
the {\em net utility}, i.e., the expected utility of the final arm selected, less the cost of gathering the evidence. 
In the classical case~\citep{Bechhofer:1954}, evidence might consist of physical samples
from a product batch; in a metalevel problem with Monte Carlo simulations,
the evidence consists of outcomes of sampling computations:

\begin{dfn} \label{dfn:metalevel-model}
	A \term{metalevel probability model} is a tuple $(U_1,\dots,U_k,\Evidence)$ 
	consisting of jointly distributed random variables:
	\begin{itemize}
		\item Real random variables $U_1,\dots,U_k$, where $U_i$ is the utility of arm~$i$, and
		\item A countable set $\Evidence$ of random variables, each variable $E\in\Evidence$ being 
		      a computation that can be performed and whose value is the result of that computation.
	\end{itemize}
\end{dfn}
%% Not actually true that we necessarily define this with U's as roots:  in sampling models they are naturally /leaves/.
%% Bounedness assumption
For simplicity, in the below we'll assume the utilities $U_i$ are bounded, 
without loss of generality in $[0,1]$.

\begin{example}[Bernoulli sampling]\label{example:bernoulli}
%% One simple metalevel probability model is used as a model of the results of simulations \citep{someone}.
In the \term{Bernoulli metalevel probability model},
each arm will either succeed or not $U_i\in\{0,1\}$, with an unknown latent frequency of success $\Theta_i$, 
and stochastic simulations of possible consequences $E_{ij}$ can be performed:
\begin{align*}
	\Theta_i &\iid {\rm Uniform}[0,1]                      &&\quad\text{for $i\in\{1,\dots,k\}$}\\
	U_i \given \Theta_i &\sample {\rm Bernoulli}(\Theta_i) &&\quad\text{for $i\in\{1,\dots,k\}$}\\
	E_{ij} \given \Theta_i &\iid {\rm Bernoulli}(\Theta_i) &&\quad\text{for $i\in\{1,\dots,k\}$, $j\in\Nat$}
\end{align*}

The \term{one-armed Bernoulli metalevel probability model} has $k=2$,
$\Theta_1=\lambda\in[0,1]$ a constant, and $\Theta_2\sample{\rm Uniform}[0,1]$.
\end{example}

A metalevel probability model, when combined with a cost of computation $c>0$,%
\footnote{The assumption of a fixed cost of computation is a simplification; 
	precise conditions for its validity are given by \citet{Harada:1997}.} 
defines a metalevel decision problem: what is the optimal strategy with which to choose a sequence 
of computations $E\in\Evidence$ in order to maximize the agent's net utility?
Intuitively, this strategy should choose the computations that give the most evidence relevant
to deciding which arm to use, stopping when the cost of computation 
outweighs the benefit gained. We formalize the selection problem as a Markov Decision Process (MDP; 
see, e.g., \citet{Puterman:1994}):

\begin{dfn}
A (countable state, undiscounted) \term{Markov Decision Process} (MDP) is a tuple $M=(S,s_0,A_s,T,R)$ where:
	$S$ is a countable set of states,
	$s_0\in S$ is the fixed initial state,
	$A_s$ is a countable set of actions available in state $s\in S$,
	$T(s,a,s')$ is the transition probability from $s\in S$ to $s'\in S$ after performing action $a\in A_s$,
	and $R(s,a,s')$ is the expected reward received on such a transition.
\end{dfn}

To formulate the metalevel decision problem as an MDP, we define the states as sequences of
computation outcomes and allow for a terminal state when the agent chooses to stop computing and act:

\begin{dfn}\label{dfn:metalevel-mdp}
	Given a metalevel probability model%
		\footnote{Definition~\ref{dfn:metalevel-model} made no assumption about the computational result
			variables $E_i\in\Evidence$, but for simplicity in the following we'll assume that
			each $E_i$ takes one of a countable set of values.  Without loss of generality, 
			we'll further assume the domains of the computational variables $E\in\Evidence$ are disjoint.}
	$(U_1,\dots,U_k,\Evidence)$ and
	a cost of computation $c>0$, a corresponding \term{metalevel decision problem}
	is any MDP $M=(S,s_0,A_s,T,R)$ such that
	\begin{align*}
		S &= \{\bot\}\cup\{\langle e_1\dots, e_n \rangle : e_i\in E_i \text{ for all $i$,} \\
								& \qquad\qquad \text{for finite $n\ge0$ and distinct $E_i\in\Evidence$}\} \\
		s_0 &= \langle \rangle \\
		A_s &= \{\bot\}\cup\Evidence_s \\
	%
	\intertext{where $\bot\in S$ is the unique terminal state,
	where $\Evidence_s\subseteq\Evidence$ is a state-dependent subset of allowed computations,
	and when given any $s=\langle e_1, \dots, e_n \rangle\in S$,
	computational action $E\in\Evidence$, 
	and $s'= \langle e_1, \dots, e_n, e \rangle\in S$ where $e\in E$, we have:}
	%
		T(s,E,s') &= P(E=e \given E_1=e_1,\dots,E_n=e_n) \\
		T(s,\bot,\bot) &= 1 \\
		R(s,E,s') &= -c \\
		R(s,\bot,\bot) &= \max_i \mu_i(s) % \max_i \IE[U_i \given E_1=e_1,\dots,E_n=e_n]
	\end{align*}		
	where $\mu_i(s) = \IE[U_i \given E_1=e_1,\dots,E_n=e_n]$.
\end{dfn}



One can optionally add an external constraint on the number of computational actions, or their total cost,
in the form of a deadline or {\em budget}.  This bridges with the related area of budgeted learning \citep{Madani+et+al:2004}.
Although this feature is not formalized in the MDP, it can be added by including either time or past total cost 
as part of the state.

\begin{example}[Bernoulli sampling]\label{example:bernoulli2}
In the Bernoulli metalevel probability model (\exampleref{example:bernoulli}),
note that: 
\begin{align}
	\Theta_i \given E_{i1},\dots,E_{in_i} &\sim {\rm Beta}(s_i+1, f_i+1)  \label{eq:bernoulli1}\\
	E_{i(n_i+1)} \given E_{i1},\dots,E_{in_i} &\sim {\rm Bernoulli}\left(\frac{s_i+1}{n_i+2}\right) \label{eq:bernoulli2} \\
	\IE[U_i \given E_{i1},\dots,E_{in_i}] &= (s_i+1)/(n_i+2) \label{eq:bernoulli3}
\end{align}
by standard properties of these distributions, where $s_i=\sum_{j=1}^{n_i} E_{in_i}$
is the number of simulated successes of arm~$i$, and $f_i=n_i-s_i$ the failures.  By \eqref{eq:bernoulli1}, 
the state space is the set of all $k$ pairs $(s_i,f_i)$; \eqrefs{eq:bernoulli2}{eq:bernoulli3}
suffice to give the transition probabilities and terminal rewards, respectively.
The one-armed Bernoulli case is similar, requiring as state just $(s,f)$ defining the posterior over $\Theta_2$.
\end{example}


Given a metalevel decision problem $M=(S,s_0,A_s,T,R)$ one defines policies and 
value functions as in any MDP.  A (deterministic, stationary) \term{metalevel policy} 
$\pi$ is a function mapping states $s\in S$ to actions to take in that state $\pi(s)\in A_s$.	

The \term{value function} for a policy $\pi$ gives the expected total reward
received under that policy starting from a given state $s\in S$, and the \term{Q-function}
does the same when starting in a state $s\in S$ and taking a given action $a\in A_s$: 
\begin{equation}
		V^\pi_M(s) = \IE^\pi_M\left[ \sum_{i=0}^N R(S_i,\pi(S_i),S_{i+1}) \given S_0 = s\right]	\label{eq:value-fn}\\
	%%	Q^\pi_M(s,a) &= \IE^\pi_M\left[ \sum_{i=0}^N R(S_i,\pi(S_i),S_{i+1}) \given S_0 = s, A_0 = a\right] \label{eq:qvalue-fn}
\end{equation}
where $N\in[0,\infty]$ is the random time the MDP is terminated, i.e.,
the unique time where $\pi(S_N)=\bot$,
%% \footnote{Whenever $N<\infty$, $S_{N+1}=\bot$ the unique terminal state.
%% Instead of having a random termination time, one can equivalently make
%% the state $\bot$ absorbing with zero reward on all transitions.}
and similarly for the Q-function $Q^\pi_M(s,a)$.

As usual, an \term{optimal policy} $\pi^*$, when it exists, is one that maximizes 
the value from every state $s\in S$, i.e., if we define for each $s\in S$
\[
	V^*_M(s)   = \sup_\pi V^\pi_M(s),
%%	Q^*_M(s,a) &= \sup_\pi Q^*_M(s,a),
\]
then an optimal policy $\pi^*$ satisfies $V^{\pi^*}_M(s) = V^*_M(s)$
for all $s\in S$, where we break ties in favor of stopping.
%% and by standard results, $\pi^*(s) \in \argmax_{a\in A_s} Q^*_M(s,a)$, where 

The optimal policy must balance the cost of computations with the improved decision
quality that results.  This tradeoff is made clear in the value function:

\begin{thm}\label{thm:value-of-computation}
	The value function of a metalevel decision process $M=(S,s_0,A_s,T,R)$ is of the form
	\[
		V^\pi_M(s) = \IE^\pi_M[ -c\,N + \max_i \mu_i(S_N) \given S_0=s]
	\]
	where $N$ denotes the (random) total number of computations performed;
	similarly for $Q^\pi_M(s,a)$.
\end{thm}

\begin{proof}
	Follows immediately from \eqref{eq:value-fn} and the definition of the
	reward function in \dfnref{dfn:metalevel-mdp}.
\end{proof}

In many problems, including the Bernoulli sampling model of \exampleref{example:bernoulli2},
the state space is infinite.  Does this preclude solving for the optimal policy?  Can 
infinitely many computations be performed?

There is in full generality an upper bound on the \emph{expected} number of computations
a policy performs:

\begin{thm}\label{thm:bounded-expected-computations}
	The optimal policy's expected number of computations is bounded by the 
	value of perfect information \citep{Howard:1966} times the inverse cost $1/c$:
	\[
		\IE^{\pi^*}[N\given S_0=s] \le \frac{1}{c} \left(\IE[\max_i U_i\given S_0=s] - \max_i \mu_i(s)\right).
	\]
	Further, any policy $\pi$ with infinite expected number of computations 
	%% $\IE^\pi[N]=\infty$
	has negative infinite value, hence the optimal
	policy stops with probability one.
\end{thm}

\begin{proof}
	The first follows as in state $s$ the optimal policy has value at least that
	of stopping immediately ($\max_i \mu_i(s)$), and as $\IE \max_i\mu_i(S_N) \le \IE \max_i U_i$ by Jensen's inequality.
	The second from \thmref{thm:value-of-computation}.
\end{proof}

Although the \emph{expected} number of computations is always bounded,
there are important cases in which the \emph{actual} number is not, such as
the following inspired by the sequential probability ratio test \citep{Wald+1945}:

\begin{example}\label{example:sprt}
Consider the Bernoulli sampling model for two arms but with a different prior:
	$\Theta_1=1/2$,
	and $\Theta_2$ is $1/3$ or $2/3$ with equal probability.
%
Simulating arm~1 gains nothing, and after $(s,f)$ simulated successes and failures of arm~2
the posterior odds ratio is
\[
	\frac{P(\Theta_2=2/3\given s,f)}{P(\Theta_2=1/3\given s,f)} = \frac{(2/3)^s(1/3)^f}{(1/3)^s(2/3)^f}= 2^{s-f}.
\]
Note that this ratio completely specifies the posterior distribution of $\Theta_2$,
and hence the distribution of the utilities and all future computations.  Thus, whether
it is optimal to continue is a function only of this ratio, and thus of $s-f$.
%% In particular,
%% note that after equal numbers of successes and failures, $s-f=0$ and so the posterior 
%% distribution over $\Theta_2$ is equal to the prior.
For sufficiently low cost, the 
optimal policy samples when $s-f$ equals $-1$, $0$, or $1$.  But with probability
$1/3$, a state with $s-f=0$ transitions to another state $s-f=0$ after two samples, 
giving finite, although exponentially decreasing, probability to arbitrarily long 
sequences of computations.
\end{example}


However, in a number of settings, including the original Bernoulli model of \exampleref{example:bernoulli},
we can prove an upper bound on the number of computations.  For reasons of space,
and for its later use in \secref{sec:blinkered}, we prove here the bound for the one-armed Bernoulli model.

Before we can do this, we need to get an analytical handle on the optimal policy.
The key is through a natural approximate policy:

\begin{dfn}\label{dfn:myopic}
	Given a metalevel decision problem $M=(S,s_0,A_s,T,R)$,
	the \term{myopic policy} $\pi^m(s)$ is defined to equal $\argmax_{a\in A_s} Q^m(s,a)$ 
	where $Q^m(s,\bot) = \max_i \mu_i(s)$ and
	\begin{equation*}%% \label{eq:myopic}
		 Q^m(s,E) = \IE_M[ -c + \max_i \mu_i(S_1) \given S_0 = s, A_0 = E].		
	\end{equation*}
\end{dfn}

The myopic policy takes the best action, to either stop or perform a computation,
under the assumption that at most one further computation can be performed.
It has a tendency to stop too early, because changing one's mind about which real action to take often takes more than one computation.
In fact, we have:

%% \note{Theorem: if Optimal stops in x, myopic stops in x (converse is more useful!)}
	
\begin{thm}\label{thm:optimal-myopic}
	Given a metalevel decision problem $M=(S,s_0,A_s,T,R)$
	if the myopic policy performs some computation in state $s\in S$,
	then the optimal policy does too, i.e., if $\pi^m(s)\neq\bot$ then $\pi^*(s)\neq\bot$.
\end{thm}

\begin{hiddenproof}
	\begin{proof}
		Observe that the myopic Q-function \eqref{dfn:myopic} is equivalently given by
		\[
			Q^m(s,a) = Q^\bot(s,a)
		\]
		where $\bot$ is the policy which immediately stops $\bot(s)=\bot$.
		Thus $Q^m(s,a) \le Q^*(s,a)$.  If the optimal policy stops in a state $s\in S$ then
		\[
			Q^{\pi^*}(s,a) \le \max_i \mu_i(s),
		\]
		and so the same holds for $Q^m$, showing the myopic stops.
	\end{proof}
\end{hiddenproof}

% Similarity to an optimal stopping result
	% Search <blah> for one-step lookahead rules.
	% But HERE is generalized to a stopping problem where there are many ways to continue.

Despite this property, the stopping behavior of the myopic policy does
have a close connection to that of the optimal policy:

\begin{dfn}\label{dfn:closed}
	Given a metalevel decision problem $M=(S,s_0,A_s,T,R)$,
	a subset $S'\subseteq S$ of states is \term{closed under transitions}
	if whenever $s'\in S'$, $a\in A_{s'}$, $s''\in S$, and $T(s',a,s'')>0$,
	we have $s''\in S'$.
\end{dfn}

\begin{thm}\label{thm:myopic-optimal}
	Given a metalevel decision problem $M=(S,s_0,A_s,T,R)$
	and a subset $S'\subseteq S$ of states closed under transitions,	
	if the myopic policy stops in all states $s'\in S'$
	then the optimal policy does too.	
\end{thm}

\begin{hiddenproof}
	\begin{proof}
	Take any $s^*\in S'$, and note that all states the chain can transition
	to from $s^*$ are also in $S'$, by transition closure.  Defining $m(s) = \max_i\mu_i(s)$, 
	observe the myopic stopping for all such states implies that
	\begin{align*}
		\IE^{\pi}[(m(S_{j+1}) - c)\, 1(j<N)\given S_0=s^*] \\
		\le \IE^{\pi}[m(S_{j})\, 1(j<N)\given S_0=s^*]
	\end{align*}
	holds for all $j$, and as a result:
	\begin{align*}
		V^\pi(s) 
		&= \IE^{\pi}[ - c N + m(S_N) \given S_0=s^*] \\
		&= \IE^{\pi}[m(S_0) + \sum_{j=0}^{N-1} (m(S_{j+1}) - c - m(S_j)) \given S_0=s^*] \\
	%%	&\le \IE^{\pi}[m(S_0) + \sum_{j=0}^{N-1} 0 |S_0=s] \\		
		&\le \max_i\mu_i(s^*) \qedhere
	\end{align*}
	\end{proof}	
\end{hiddenproof}

Using this, we can prove our bound:

\begin{thm}\label{thm:one-action-bound}
	The one-armed Bernoulli decision process with constant arm $\lambda\in[0,1]$ 
	performs at most $\lambda(1-\lambda)/c-3 \le 1/4c-3$ computations.
\end{thm}
\begin{proof}
Using \dfnref{dfn:myopic} and \exampleref{example:bernoulli2}, we determine
which states the myopic policy stops in by bounding $Q^m(s,E)$.  For a state $(s,f)$,
let $\mu=(s+1)/(n+2)$ be the mean utility for arm~2, where $n=s+f$.
Fixing $n$ and maximizing over $\mu$, we get sufficient condition for stopping
\begin{equation}
	c \ge \frac{\lambda(1-\lambda)}{(n+3)} \qquad\qquad   n\ge \frac{\lambda(1-\lambda)}{c} - 3  \label{eq:myopic-stopping}
\end{equation}
Since the set of states satisfying \eqref{eq:myopic-stopping} is closed under
transitions ($n$ only increases), by \thmref{thm:optimal-myopic}.  Finally, note $\max_{\lambda\in[0,1]} \lambda(1-\lambda)=1/4$.
\end{proof}	

\begin{hiddenproof}
	\begin{proof}
		By \dfnref{dfn:myopic} and \exampleref{example:bernoulli2}, the myopic policy stops in
		a state $(s,f)$ when
		\begin{align}
			c \ge &\mu\max(\mu^+,m) + (1-\mu_i)\max(\mu^-, m) - \max(\mu,m) \label{eq:stopping}
		\end{align}
		where $\mu=(s+1)/(n+2)$  is the mean utility for arm~2, where $n=s+f$,
		$\mu^- = \mu - \mu/(n+3)$, and $\mu^+ = \mu + (1-\mu)/(n+3)$
		are the posterior means of arm~2 after simulating a failure and a success,
		%% and $\mu^-_i=(s_i+1)/(s_i+f_i+3) $ and $\mu^+_i = (s_i+2)/(s_i+f_i+3)$
		%% are the posterior means of action $i$ after simulating a failure and a success,
		respectively.  Whenever \eqref{eq:stopping} holds, stopping is preferred to sampling.

		Fixing $n$ and maximizing over $\mu$, we get sufficient condition for stopping
		\begin{equation}
			c \ge \frac{\lambda(1-\lambda)}{(n+3)} \qquad\qquad   n\ge \frac{\lambda(1-\lambda)}{c} - 3  \label{eq:myopic-stopping}
		\end{equation}
		Since the set of states satisfying \eqref{eq:myopic-stopping} is closed under
		transitions ($n$ only increases), by \thmref{thm:optimal-myopic}.  Finally, note $\max_{\lambda\in[0,1]} \lambda(1-\lambda)=1/4$.
	\end{proof}	
\end{hiddenproof}

A key implication is that the \emph{optimal} policy can be computed
in time $O(1/c^2)$, i.e., quadratic in the inverse cost.  This is particularly appropriate when the cost of 
computation is relatively high, such as in simulation experiments \citep{Swisher+et+al:2003},
or when the decision to be made is critical. Finite bounds for the $k$-armed problem
can also be derived and will be included in the full paper.
